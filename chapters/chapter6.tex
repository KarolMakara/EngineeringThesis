

\chapter{Conclusions}
\label{char:conclusions}

In conclusion, the evaluation of both egress and ingress scenarios highlights the strengths of Antrea and Cilium CNI plugins. In the egress scenario, Antrea demonstrates lower resource usage, with higher throughput and lower round-trip time. However, Cilium's egress gateway implementation offers more control over routing outgoing traffic, with additional options for matching traffic to flow through the gateway. Both CNIs provide high availability for the egress gateway, though Cilium requires an enterprise version of the plugin. For the ingress scenario, Cilium consumes fewer resources in a local environment where Kubernetes nodes are created as Docker containers. On the other hand, Antrea, when combined with NGINX, exhibits lower resource usage in cloud environments. Cilium, however, achieves better traffic weighting accuracy (with some statistical uncertainty) compared to Antrea in both environments.

 

Overall, Cilium offers greater configurability and a robust set of features, including its own Gateway API and Egress Gateway implementation. Additionally, it supports eBPF, which can provide significant advantages in larger clusters. In contrast, the Antrea CNI plugin offers an efficient Egress Gateway with lower resource usage, making it a better option for smaller clusters with limited resources, where the need for both a Gateway API and an Egress Gateway provided by a single CNI may not be necessary.

Future work could focus on evaluating the performance and resource utilization of Antrea and Cilium in more complex and larger clusters. As noted by the authors of Cilium, eBPF offers greater efficiency in large-scale environments. Expanding future scenarios to include multiple Gateway APIs and egress gateways, with traffic generated by thousands or even tens of thousands of pods, would provide deeper insights into the performance of container networking plugins under highly demanding conditions.