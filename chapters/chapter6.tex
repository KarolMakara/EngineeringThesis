

\chapter{Conclusions}
\label{char:conclusions}

In conclusion, the evaluation of egress and ingress scenarios highlights strengths of Antrea and Cilium CNI plugins. In the egress scenario, despite Antrea demonstrating lower resource usage, the throughput is higher and round-trip time is lower. However, Cilium's egress gateway implementation provides better control over routing outgoing traffic by offering more options to match traffic to flow through the gateway. Both CNIs provide high availability egress gateway, but Cilium requires an enterprise version of the plugin. For ingress scenario, Cilium performs less resource consumption in local environment, where Kubernetes nodes are created as docker containers. Antrea with NIGNX has lower resource usage in cloud environments, where Cilium achieves better traffic weighting accuracy (with statistical uncertainty). 

 

Overall, Cilium offers greater configurability and robust features, including its own Gateway API and Egress Gateway implementation. It also offers support for eBPF feature which might get advantage in larger clusters. Antrea CNI plugin provides own Egress Gateway with efficient resource usage, which might be better option for smaller clusters, where resources are limited, and the cluster does not require both Gateway API and Egress Gateway provided by single container network interface. 

 

Future work can focus on evaluating the performance and resource use of Antrea and Cilium plugins in more complex and larger clusters. According to the authors of Cilium, eBPF demonstrates greater efficiency in large-scale environments. Future scenarios could be expanded to use multiple Gateway APIs and egress gateways, with traffic generated by thousands or even tens of thousands of pods. This would provide deeper insights into the performance of container networking plugins under highly demanding conditions. 