\chapter{Background and Related Work}
\label{cha:background}


%---------------------------------------------------------------------------

\section{Containerization}
\label{sec:containerization}

Contenerization is packaging an app along with all necessary runtime stuff like libraries, executables or asets into an object called "container". The main benefits of container are\cite{RedhatContainerization}:
\begin{itemize}
    \item Protable and Flexible -- container can be run on bare metal or virtual machine in cloud regardless of operating system. Only a container runtime software like \href{https://docs.docker.com/engine/}{Docker Engine} or \href{https://containerd.io/}{containerd} is required, which allows to interact with host system.
    \item Lightweight -- container is sharing operating system kernel with hostmachine, there is no need to install separate operating system inside
    \item Isolated -- does not depends on host's environment or infrastructure
    \item Standarized -- \href{https://opencontainers.org/}{Open Container Initiative} standarize runtime, image and distribution specifications
\end{itemize}
A container image is set of files and configuration needed to run a container. It is immutable, only new image can be crated with new changes. Consists of layers. The layer contain one modification made a image. All layers are cachable and can be reused when building an image. The mechanism is really usefull when compiling large application components inside one container\cite{DockerImage}.

%---------------------------------------------------------------------------

\section{Contianer Orchestration}
\label{sec:ContianerOrchestration}

Container orchestration is coordinated deploying, managing, networking, scaling and monitoring containers process. It automates and manages whole container's lifecycle, there is no need to worrying about of deployed app, orchestration software like \href{https://kubernetes.io/}{Kubernetes} will take care of its availability \cite{RedhatContainerization}.

The Kubernetes Authors says: "The name Kubernetes originates from Greek, meaning helmsman or pilot. K8s as an abbreviation results from counting the eight letters between the "K" and the "s"". K8s is open-source orchestration platform capable of managing containers. Key functionalities are \cite{KubernetesDocs}:

\begin{itemize}
    \item Automated rollouts and rollbacks -- updates or downgrades version of deplotyed containers at controller rate, replacing containers incrementally
    \item Automatic bin packing --  allows to specify exact resources needed by container (CPU, Memory) to fit on appropriate node
    \item Batch execution -- possible to create sets of tasks which can be run without manual intervention
    \item Designed for extensibility -- permits to add feautres using custom resource definitions without changing source code
    \item Horizontal scaling -- scales (replicate) app based of its need for resources
    \item IPv4/IPv6 dual-stack -- allocates IPv4 or IPv6 to pods and services
    \item Secret and configuration management -- allows store, manage and update secrets. Containers do not have to be rebuilt to access updated  credentials
    \item Self-healing -- restarts crashed containers or by failure specified by user
    \item Service discovery and load balancing -- advertises a container using DNS name or ip. Loadbalances traffic across all pods in deployment
    \item Storage orchestration -- mounts desired storage like local or shipped by cloud provider
\end{itemize}



%---------------------------------------------------------------------------

\section{Kubernetes Architecture}
\label{sec:k8s_arch}
A cluster is set of machines controlled by K8s.
Tekst:
\begin{enumerate}%[1)]
\item Tekst

\item Tekst
\end{enumerate}


\begin{lstlisting}[language=Python, frame=single, caption={Python function to greet}]
    def greet(name):
        print(f"Hello, {name}!")
\end{lstlisting}

%---------------------------------------------------------------------------

\subsection{Kuberenetes Basics}
\label{sec:k8s_basics}
%---------------------------------------------------------------------------

\subsection{Control Plane}
\label{sec:k8s_cplane}
%---------------------------------------------------------------------------

\subsection{Nodes}
\label{sec:k8s_nodes}
%---------------------------------------------------------------------------

\subsection{Objects}    
\label{sec:k8s_objects}
%---------------------------------------------------------------------------

\subsection{Interfaces}    
\label{sec:k8s_objects}

\subsubsection{TEST}
TEKST asdasdasda
%---------------------------------------------------------------------------



\section{Cluster Networking}
\label{sec:k8s_networking}

Tekst
\begin{enumerate}%[1)]
\item Tekst

\item Tekst
\end{enumerate}
Tekst

%---------------------------------------------------------------------------

\section{Container Network Interface (CNI)}
\label{sec:cni_intro}




%---------------------------------------------------------------------------

\section{Overview of Selected CNI Plugins}
\label{sec:cni_overview}
%---------------------------------------------------------------------------

\section{Related Work}
\label{sec:realted_work}


DoTekst
Na stronie \underline{\texttt{http://kile.sourceforge.net/screenshots.php}} Tekst {\em Kile}, Tekst

Tekst

\begin{itemize}
\item Tekst
\end{itemize}
