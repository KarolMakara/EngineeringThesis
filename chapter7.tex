\chapter{Ingress Scenario Comparison}
\label{cha:ingress_compare}

W rozdziale tym przedstawiono podstawowe informacje dotyczące struktury prostych plików \LaTeX a. Omówiono również metody kompilacji plików z zastosowaniem programów \emph{latex} oraz \emph{pdflatex}.

%---------------------------------------------------------------------------

\section{Resource Utilization}
\label{sec:traffic_concept}

Plik \LaTeX owy jest plikiem tekstowym, który oprócz tekstu zawiera polecenia formatujące ten tekst (analogicznie do języka HTML). Plik składa się z dwóch części:
\begin{enumerate}%[1)]
\item Preambuły -- określającej klasę dokumentu oraz zawierającej m.in. polecenia dołączającej dodatkowe pakiety;

\item Części głównej -- zawierającej zasadniczą treść dokumentu.
\end{enumerate}


\begin{lstlisting}
\documentclass[a4paper,12pt]{article}      % preambuła
\usepackage[polish]{babel}
\usepackage[utf8]{inputenc}
\usepackage[T1]{fontenc}

\begin{document}                           % część główna

\section{Sztuczne życie}

% treść
% ąśężźćńłóĘŚĄŻŹĆŃÓŁ

\end{document}
\end{lstlisting}

Nie ma żadnych przeciwskazań do tworzenia dokumentów w~\LaTeX u w~języku polskim. Plik źródłowy jest zwykłym plikiem tekstowym i~do jego przygotowania można użyć dowolnego edytora tekstów, a~polskie znaki wprowadzać używając prawego klawisza \texttt{Alt}. Jeżeli po kompilacji dokumentu polskie znaki nie są wyświetlane poprawnie, to na 95\% źle określono sposób kodowania znaków (należy zmienić opcje wykorzystywanych pakietów).


%---------------------------------------------------------------------------

\section{Traffic Splitting}
\label{sec:traffic_mngmnt}


Załóżmy, że przygotowany przez nas dokument zapisany jest w pliku \texttt{test.tex}. Kolejno wykonane poniższe polecenia (pod warunkiem, że w pierwszym przypadku nie wykryto błędów i kompilacja zakończyła się sukcesem) pozwalają uzyskać nasz dokument w formacie pdf:
\begin{lstlisting}
latex test.tex
dvips test.dvi -o test.ps
ps2pdf test.ps
\end{lstlisting}
%
lub za pomocą PDF\LaTeX:
\begin{lstlisting}
pdflatex test.tex
\end{lstlisting}

Przy pierwszej kompilacji po zmiane tekstu, dodaniu nowych etykiet itp., \LaTeX~tworzy sobie spis rozdziałów, obrazków, tabel itp., a dopiero przy następnej kompilacji korzysta z tych informacji.

W pierwszym przypadku rysunki powinny być przygotowane w~formacie eps, a~w~drugim w~formacie pdf. Ponadto, jeżeli używamy polecenia \texttt{pdflatex test.tex} można wstawiać grafikę bitową (np. w formacie jpg).


\subsection{Virtual Users}
\label{sec:vus}

\subsection{Time Window}
\label{sec:time_wind}

